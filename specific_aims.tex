Furthering our understanding of how ``mental representations'' translate into ``neural representations'' is integral to establishing links between theoretical cognitive models and the functional realities of the human brain. This is one of the central aims of cognitive neuroscience, with broad applications throughout health sciences. A primary tool for investigating the mind--brain connection is fMRI, and it has been applied pervasively to shed light primarily on {\em where} various processes and representations reside in the brain. Although most typically used for localizing, the technology and the methods for analyzing fMRI data have developed to a point where more specific questions about {\em how} the brain represents the content of our mental representations can be addressed. 

Critically, {\em where} and {\em how} the brain represents the content of mental representations are related in at least two ways. The first is that, because the roles of some anatomic regions of the brain have been well established, more complex cognitive tasks may appear to decompose to a set of simpler ones based on where the activation is discovered. The second has to do directly with how the question of {\em where} is addressed. All methods for identifying interesting signal in fMRI datasets must make some {\it a priori} assumptions about what such signal is expected to look like. This is necessary because of the sheer size of the datasets and the challenge of weeding out spurious effects. As a consequence, aspects of the data that violate these assumptions will be, for better and for worse, ignored. Because the expected answer to {\em how} bears directly on these assumptions, {\em where} and {\em how} are intrinsically correlated in ways that are not obvious but highly influential (Cox and Rogers, submitted). 

This may speak to the inconclusiveness in the literature regarding how even comparitively simple concepts---concrete nouns---are neurally represented. There simultaneously exists evidence for functionally specific brain regions that handle different kinds of concepts (e.g., Kanwisher; Mahon), local regions of richly informative distributed representations that neurally represent a wide range of conceptual domains (e.g. Haxby; others), and for widely distributed networks where concepts are dispersed throughout the brain, primarily across regions of the brain closely related to perception and action (XXX). There is some evidence for modality dependence: the concept [LABRADOR] presented as either a picture or a word may be associated with different neural representations (XXX). More complex concepts, such as emotions and other abstract nouns, are even less well understood.


